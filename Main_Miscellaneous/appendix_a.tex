% !TeX root = ../main.tex
\chapter{定理5.1证明}


为了证明\textbf{定理 5.1},该附录首先介绍互信息的相关性质。具体来说,
% \subsection{互信息的性质} 
对于任意的随机变量$\Xmat$、 $\Ymat$ 和 $\Zmat$:
\begin{enumerate}[itemindent=1.2em]
\item[$(P_1)$] 非负性:$I(\Xmat; \Ymat) \ge 0, ~I(\Xmat; \Ymat|\Zmat)\ge 0$;
% \begin{align*}
% \end{align*}
\item[$(P_2)$] 链式法则:
\begin{align*}
I(\Xmat, \Ymat; \Zmat) &= I(\Ymat; \Zmat) + I(\Xmat; \Zmat|\Ymat) = I(\Xmat; \Zmat) + I(\Ymat; \Zmat|\Xmat)\\
&= \frac{1}{2} [I(\Ymat; \Zmat) + I(\Xmat; \Zmat) + I(\Xmat; \Zmat|\Ymat) + I(\Ymat; \Zmat|\Xmat)];
\end{align*}
\item[$(P_3)$] 链式法则(多元互信息): $I(\Xmat; \Ymat; \Zmat) = I(\Ymat; \Zmat) - I(\Ymat; \Zmat|\Xmat)$;
% \begin{align*}
% I(\Xmat; \Ymat; \Zmat) = I(\Ymat; \Zmat) - I(\Ymat; \Zmat|\Xmat). 
% \end{align*}
\item[$(P_4)$] 离散熵的非负性(离散变量 $\Xmat$): $H(\Xmat)\ge 0, H(\Xmat|\Ymat)\ge 0$;
% \begin{align*}
% H(\Xmat)\ge 0, H(\Xmat|\Ymat)\ge 0. 
% \end{align*}
\item[$(P_5)$] 互信息与熵:$H(\Xmat)= H(\Xmat|\Ymat)+ I(\Xmat; \Ymat)$。
% \begin{align*}
% H(\Xmat)= H(\Xmat|\Ymat)+ I(\Xmat; \Ymat). 
% \end{align*}
\end{enumerate}

% *******************************************************************
% \subsection{引理声明} 
然后,本文声明三个易证的引理。

\begin{lemma}
\label{lem:mi_xz_condition}
在表征学习过程中,给定随机变量 $\Xmat$,定义随机变量 $\Zmat$ 为$\Xmat$对应的表征;一旦$\Xmat$被观测,则$\Zmat$条件独立于模型中任何其它变量。
即,对于模型中任意其它变量$\Tmat_1$ 和 $\Tmat_2$,有
\begin{align}
I(\Zmat; \Tmat_1|\Xmat, \Tmat_2) = 0. 
\end{align}
\end{lemma} 


% *******************************************************************
\begin{lemma}
\label{lem:mi_xz_mutual}
给定随机变量 $\Xmat_1,$ $\Xmat_2, ..., \Xmat_n$ 和确定性函数 $f$,对 $\forall \, i, j = 1, 2, ..., n,$ 有
\begin{align}
I(\Xmat_i; f(\Xmat_i)) \ge I(\Xmat_j; f(\Xmat_i)). 
\end{align}
\end{lemma}
\begin{proof}
根据定义有,
\begin{align*}
I(\Xmat_i; f(\Xmat_i)) &= H(f(\Xmat_i)) - H(f(\Xmat_i) \mid \Xmat_i), \\
I(\Xmat_j; f(\Xmat_i)) &= H(f(\Xmat_i)) - H(f(\Xmat_i) \mid \Xmat_j). 
\end{align*}
$f$ 是确定性函数,所以
\begin{align*}
H(f(\Xmat_i) \mid \Xmat_i) &= 0, ~~H(f(\Xmat_i) \mid \Xmat_j) \ge 0. 
\end{align*}
因此,
\begin{align*}
I(\Xmat_i; f(\Xmat_i)) \ge I(\Xmat_j; f(\Xmat_i)).
\end{align*}
\end{proof}


% *******************************************************************
\begin{lemma}
分别用 $\Zmat_1$ 和 $\Zmat_2$ 表示 $\Xmat_1$ 和 $\Xmat_2$ 的表征,则
\begin{align}
&I_\theta(\Xmat_1;\Zmat_1|\Xmat_2) \le \KL(p_\theta(\Zmat_1|\Xmat_1)||p_\psi(\Zmat_2|\Xmat_2)), \\
&I_\psi(\Xmat_2;\Zmat_2|\Xmat_1) \le \KL(p_\psi(\Zmat_2|\Xmat_2)||p_\theta(\Zmat_1|\Xmat_1)). 
\end{align} 
\end{lemma} 
\begin{proof}
根据定义有,
\begin{align*}
I_\theta(\Xmat_1; \Zmat_1|\Xmat_2)
&= \E_{\xv_1, \xv_2 \sim p(\Xmat_1, \Xmat_2)} \E_{\zv \sim p_\theta(\Zmat_1|\Xmat_1)}\left[\log\frac{p_\theta(\Zmat_1 = \zv|\Xmat_1=\xv_1)}{p_\theta(\Zmat_1=\zv|\Xmat_2=\xv_2)}\right]\\
&=\E_{\xv_1, \xv_2\sim p(\Xmat_1, \Xmat_2)}\E_{\zv \sim p_\theta(\Zmat_1|\Xmat_1)}\left[\log\frac{p_\theta(\Zmat_1=\zv | \Xmat_1=\xv_1)}{p_\psi(\Zmat_2 = \zv | \Xmat_2 = \xv_2)}
\right] \\ 
&\quad - \E_{\xv_1, \xv_2 \sim p(\Xmat_1, \Xmat_2)}\E_{\zv \sim p_\theta(\Zmat_1|\Xmat_1)}\left[\log\frac{p_\theta(\Zmat_1 = \zv | \Xmat_2=\xv_2)}{p_\psi(\Zmat_2=\zv | \Xmat_2= \xv_2)}
\right]
\\ 
&=\KL(p_\theta(\Zmat_1|\Xmat_1)||p_\psi(\Zmat_2|\Xmat_2)) - \KL(p_\theta(\Zmat_2|\Xmat_1)||p_\psi(\Zmat_2|\Xmat_2))\\ 
&\le \KL(p_\theta(\Zmat_1|\Xmat_1)||p_\psi(\Zmat_2|\Xmat_2)). 
\end{align*} 
当且仅当 $p_\psi(\Zmat_2|\Xmat_2)$ 与 $p_\theta(\Zmat_1|\Xmat_2)$ 同分布时,该不等式的等号成立。类似地,本文还可以推导出 $I_\psi(\Xmat_2; \Zmat_2|\Xmat_1)\le \KL(p_\psi(\Zmat_2|\Xmat_2)||p_\theta(\Zmat_1|\Xmat_1))$。
\end{proof}


% *****************************************************************
% \section{定理 5.1 的证明}
最后,本文使用上述互信息特性和引理证明\textbf{定理 5.1}。

\begin{theorem}[互信息$I(\Xmat^v, \Xmat^l; \Tmat^v, \Tmat^l)$的上界]
给定两组随机变量$\Xmat=[\Xmat^v, \Xmat^l]$和$\Tmat=[\Tmat^v, \Tmat^l]$,互信息$I(\Xmat^v, \Xmat^l; \Tmat^v, \Tmat^l)$的下界可表述为:
\begin{equation}
\begin{aligned}
I(\Xmat; \Tmat) = I(\Xmat^v, \Xmat^l; \Tmat^v, \Tmat^l) \le \underbrace{I(\Xmat^v; \Tmat^v)}_{\text{\small{\ding{202}}}} + \underbrace{I(\Xmat^l; \Tmat^l)}_{\text{\small{\ding{203}}}} \underbrace{- I(\Tmat^v; \Tmat^l)}_{\text{\small{\ding{204}}}} + \underbrace{\SKL}_{\text{\small{\ding{205}}}}, 
\end{aligned}
\label{eq:JMI}
\end{equation}
式中,$\SKL$表示对称KL散度,可以通过平均两个KL散度$\KL(p(t^v|x^v)||p(t^l|x^l))$和$\KL(p(t^l|x^l)||p(t^v|x^v))$来计算。
\end{theorem}
\begin{proof}
\begin{align*}
I(\Xmat; \Tmat) &= I(\Xmat^l, \Xmat^v; \Tmat) \\
&\stackrel{(P_2)}{=}\frac{1}{2}[I(\Xmat^l; \Tmat) + I(\Xmat^v; \Tmat) + I(\Xmat^l; \Tmat|\Xmat^v) + I(\Xmat^v; \Tmat|\Xmat^l)]\\
&=\frac{1}{2}[I(\Xmat^l; \Tmat^l, \Tmat^v) + I(\Xmat^v; \Tmat^l, \Tmat^v) + I(\Xmat^l; \Tmat|\Xmat^v) + I(\Xmat^v; \Tmat|\Xmat^l)]
\end{align*}
其中,
\begin{align*}
I(\Xmat^l; \Tmat^l, \Tmat^v) &\stackrel{(P_2)}{=}I(\Xmat^l; \Tmat^l) + I(\Xmat^l; \Tmat^v|\Tmat^l) \\ 
&\stackrel{(P_3)}{=}I(\Xmat^l; \Tmat^l) + I(\Xmat^l; \Tmat^v) - I(\Xmat^l; \Tmat^v; \Tmat^l)\\
&\stackrel{(P_3)}{=}I(\Xmat^l; \Tmat^l) + I(\Xmat^l; \Tmat^v) - I(\Tmat^l; \Tmat^v) + I(\Tmat^l; \Tmat^v|\Xmat^l) \\
&\stackrel{(LA.1.)}{=}I(\Xmat^l; \Tmat^l) + I(\Xmat^l; \Tmat^v) - I(\Tmat^l; \Tmat^v)\\ 
&\stackrel{(LA.2.)}{\le}2I(\Xmat^l; \Tmat^l) - I(\Tmat^l; \Tmat^v). 
\end{align*}
类似地,$I(\Xmat^v; \Tmat^l, \Tmat^v)$ 的上界为: 
\begin{align*}
I(\Xmat^v; \Tmat^l, \Tmat^v) \le 2I(\Xmat^v; \Tmat^v) - I(\Tmat^l; \Tmat^v). 
\end{align*} 
又, 
\begin{align*}
I(\Xmat^l; \Tmat|\Xmat^v) &= I(\Xmat^l; \Tmat^l, \Tmat^v|\Xmat^v)\\
&\stackrel{(P_2)}{=} I(\Xmat^l; \Tmat^l|\Xmat^v) + I(\Xmat^l; \Tmat^v|\Xmat^v,\Tmat^l) \\ 
&\stackrel{(LA.1.)}{=} I(\Xmat^l; \Tmat^l|\Xmat^v), \\ 
% 
I(\Xmat^v; \Tmat|\Xmat^l) &= I(\Xmat^v; \Tmat^l, \Tmat^v|\Xmat^l)\\
&\stackrel{(P_2)}{=} I(\Xmat^v; \Tmat^v|\Xmat^l) + I(\Xmat^v; \Tmat^l|\Xmat^l,\Tmat^v) \\ 
&\stackrel{(LA.1.)}{=} I(\Xmat^v; \Tmat^v|\Xmat^l). \\ 
\end{align*}
令 $\SKL=\frac{1}{2}\left(\KL(p_\theta||p_\psi) + \KL(p_\psi||p_\theta)\right)$, \\ 
则,
\begin{align*}
I(\Xmat; \Tmat) &= I(\Xmat^l, \Xmat^v; \Tmat^l, \Tmat^v)\\
&\le I(\Xmat^l; \Tmat^l) + I(\Xmat^v; \Tmat^v) - I(\Tmat^l; \Tmat^v) + \frac{1}{2}[I(\Xmat^l; \Tmat^l|\Xmat^v) + I(\Xmat^v; \Tmat^v|\Xmat^l)]\\ 
&\stackrel{(LA.3.)}{\le} I(\Xmat^l; \Tmat^l) + I(\Xmat^v; \Tmat^v) - I(\Tmat^l; \Tmat^v)\\
&\qquad + \frac{1}{2}[\KL(p_\theta(\Tmat_1|\Xmat_1)||p_\psi(\Tmat_2|\Xmat_2)) + \KL(p_\psi(\Tmat_2|\Xmat_2)||p_\theta(\Tmat_1|\Xmat_1))]\\
&=I(\Xmat^l; \Tmat^l) + I(\Xmat^v; \Tmat^v) - I(\Tmat^l; \Tmat^v) + \SKL. 
\end{align*}
\end{proof}

