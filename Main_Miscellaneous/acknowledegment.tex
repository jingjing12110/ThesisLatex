% !TeX root = ../main.tex
% 致谢中主要感谢导师和对论文工作有直接贡献和帮助的人士和单位。
% 一般致谢的内容有:
% \begin{enumerate}[label=(\chinese*),itemindent=2em]

%     \item 对指导或协助指导完成论文的导师;
%     \item 对国家科学基金、资助研究工作的奖学金基金、合同单位、资助或支持的企业、组织或个人;
%     \item 对协助完成研究工作和提供便利条件的组织或个人;
%     \item 对在研究工作中提出建议和提供帮助的人;
%     \item 对给予转载和引用权的资料、图片、文献、研究思想和设想的所有者;
%     \item 对其他应感谢的组织和个人。

% \end{enumerate}

% % 致谢言语应谦虚诚恳,实事求是。字数不超过1000汉字


衷心感谢我的导师郑南宁教授。感谢郑老师在我科研、学习和生活上给予的指导和关心。
郑老师在学术领域的前瞻性和思考问题的高度为我的科研提供了极大的指导和帮助;
郑老师严谨的治学理念和学术态度为我树立了科研标杆,是我一生学习的榜样;
郑老师勤勉务实的工作作风激励我脚踏实地好好科研。
在博士求学期间,郑老师创建了优越的实验室环境,提供了良好的科研条件和丰富的交流机会,让我在良好的硬件环境以及和谐的学术氛围中进行科研工作。
% 感谢郑老师创建优越的实验室环境,让我在良好的硬件环境以和谐的学术氛围中进行科研工作。
% 在博士求学期间,郑老师对我的学术指导和帮助让我受益良多。


感谢交大和人机所提供的优良学术环境、实验平台及软硬件支持。

感谢人机所其他老师的帮助和指导。
感谢张雪涛老师的指导,雪涛老师在本科毕设、研究生规划及转博等方面给予了重要帮助。
感谢南智雄老师在小组讨论中提供的指导以及研究方向确定时提供的帮助。

感谢课题组同学对我的帮助和包容。感谢刘子熠师兄、陈仕韬师兄、石培文师兄、马永强师兄、张乐毅师兄、余思雨师姐、王颖师姐自本科毕设起对我的科研指导和帮助。
感谢杨奔、陈辉、彭极智、刘一凡、吕鑫、杨卓、董金鹏等同学的小组交流与讨论。
感谢实验室和课题组所有同学的共同努力。

% 感谢维檬同学。感谢你对科研的热情和纯粹,是你的坚持坚定了我的坚持。
% 你对科研的热情和纯粹让我十分钦佩和羡慕,是你的坚持坚定了我的坚持。

% 感谢小伙伴们的鼓励和陪伴。

感谢父母和家人对我的养育和照顾。感谢你们始终如一的支持和鼓励,在我情绪低落时给予我安慰,在我彷徨无助时给予我鼓励。你们的付出激励我不断前行。

最后,感谢所有关心和帮助过我的人。


% \vspace{\baselineskip}
% {\color{red} 用于双盲评审的论文,此页内容全部隐去。}

