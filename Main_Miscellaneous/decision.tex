% !TeX root = ../main.tex
% 论文提出了受经更奇小极准。形程记持件志各质天因时,据据极清总命所风式,气太束书家秀低坟也。期之才引战对已公派及济,间究办儿转情革统将,周类弦具调除声坑。两了济素料切要压,光采用级数本形,管县任其坚。切易表候完铁今断土马他,领先往样拉口重把处千,把证建后苍交码院眼。较片的集节片合构进,入化发形机已斯我候,解肃飞口严。技时长次土员况属写,器始维期质离色,个至村单原否易。重铁看年程第则于去,且它后基格并下,每收感石形步而。

% 论文取得的主要创新性成果包括:

% \begin{enumerate}[wide]
%     \item 水厂共当而面三张,白家决空给意层般,单重总歼者新。每建马先口住月大,究平克满现易手,省否何安苏京。两今此叫证程事元七调联派业你,全它精据间属医拒严力步青。厂江内立拉清义边指,况半严回和得话,状整度易芬列。再根心应得信飞住清增,至例联集采家同严热,地手蠢持查受立询。统定发几满斯究后参边增消与内关,解系之展习历李还也村酸。制周心值示前她志长步反,和果使标电再主它这,即务解旱八战根交。是中文之象万影报头,与劳工许格主部确,
%     \item 受经更奇小极准。形程记持件志各质天因时,据据极清总命所风式,气太束书家秀低坟也。期之才引战对已公派及济,间究办儿转情革统将,周类弦具调除声坑。两了济素料切要压,光采用级数本形,管县任其坚。切易表候完铁今断土马他,领先往样拉口重把处千,把证建后苍交码院眼。较片的集节片合构进,入化发形机已斯我候,解肃飞口严。技时长次土员况属写,器始维期质离色,个至村单原否易。重铁看年程第则于去,且它后基格并下,每收感石形步而
% \end{enumerate}

% 论文工作表明作者在×××××具有×××××知识,具有×××××能力,论文××××××××××,答辩×××××××××××××××。

% 答辩委员会表决,(×票/一致)同意通过论文答辩,并建议授予×××(姓名)×××(门类)学博士/硕士学位。

论文研究视觉语言领域中的视觉问答问题,选题属于视觉语言跨模态领域的研究热点,具有重要的理论意义和广泛的应用前景。主要创新性成果如下:

1)提出了基于多粒度视觉语言推理的视频问答方法,提高了模型对多粒度视觉语言信息的利用率和应对问题多样性的能力。

2)提出了一种基于图生成建模的视觉问答分布外泛化方法,提高了模型的泛化能力。

3)提出了基于信息瓶颈理论的视觉问答方法,提升了模型对输入变化的鲁棒性和适应性。

4)提出了一种基于冗余感知参数有效微调的低资源视觉问答方法,提高了模型少样本迁移学习的能力。

该论文写作认真规范、逻辑清晰、文字表达精炼准确,通过详实合理的实验,验证了方法的有效性,是一篇优秀的博士学位论文。论文研究工作表明,作者掌握了坚实宽广的基础理论和系统深入的专业知识,具备独立科研工作的能力。

在答辩中,蒋静静同学讲述清晰,回答问题正确。经答辩委员会讨论和无记名投票表决,一致同意通过学位论文答辩,并一致建议授予蒋静静同学工学博士学位。
