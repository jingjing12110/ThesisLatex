
% !TeX root = ../main.tex
\begin{chineseabstract}
% \footnotetext{*本研究得到某某基金(编号:)的资助}
% \astfootnote{本研究得到某某基金(编号:)的资助}

视觉问答是计算机视觉和自然语言处理交叉领域重要任务之一,旨在根据对图像或视频内容的理解,回答以自然语言形式给定的问题。视觉问答本质上是一个复杂场景下多学科交叉的综合问题,因其将目标识别与检测、跨模态检索、空间及常识推理、语言生成等技术统一到一个问题中而被视作一种视觉图灵测试。此外,视觉问答还是面向用户的复杂人机交互系统中的重要组成部分,具有广阔的实际应用前景,如视障人群视觉辅助、医疗问诊、安全监控、商品图文理解、智能客服及广告生成等。因具有重要的学术研究价值及广阔的实际应用前景,视觉问答成为了近几年的研究热点。本文从促进视觉问答系统实用性的角度出发,分析并总结了视觉问答面临的挑战:对视频中多粒度视觉语言信息利用不充分、语言偏见导致的泛化能力不足、对视觉语言输入变化的鲁棒性不足、低资源场景下学习能力不足,并为之提出了对应的解决方案,具体的研究内容和贡献如下:

% ,leftmargin=0pt,itemsep=1pt
\begin{enumerate}[wide,]
\item 针对视觉问答模型对多粒度视觉语言信息利用不充分这一挑战,本文提出了基于多粒度视觉语言推理的视频问答方法。该方法利用图网络分别对视觉和语言输入进行多粒度的编码,并使用多样性感知的视觉语言联合推理模块对获得的多粒度多来源表征进行有效融合。在保持表征判别性的同时自适应地学习不同表征的重要性,提高了跨模态表征整合的有效性及对多粒度视觉语言信息的利用率。实验结果表明该框架在具有最少模型参数量的条件下,取得了多个视频问答数据集上最佳的性能。


\item 针对语言偏见导致视觉问答模型泛化能力不足这一挑战,本文提出了基于图生成建模的视觉问答分布外泛化方法。
该方法将视觉问答的分布外泛化问题表述为组合泛化问题,利用域内数据来生成已有视觉概念的新组合来促使视觉问答模型生成分布外的答案,并使用梯度分布一致性损失来约束具有对抗性扰动的数据和生成的数据之间的梯度一致性,极大地缓解图生成过程中不稳定梯度的问题。
在视觉问答分布外泛化评测数据集上的实验表明该方法能显著提高基线视觉问答模型分布外泛化的能力。


\item 针对视觉问答模型对视觉语言输入变化的鲁棒性不足这一挑战,本文提出了基于相关信息瓶颈理论的鲁棒视觉问答方法。
该方法旨在减少视觉语言预训练模型生成的表征中与任务无关的冗余信息,鼓励预训练模型学习到的表征收敛到最小充分统计量,使获得的表征更加紧凑和更加任务相关。
在多个公开数据集上的大量实验表明所提方法能显著提高基线视觉问答模型的输入鲁棒性。


\item 针对视觉问答模型在低资源场景下学习能力不足这一挑战,本文提出了基于冗余感知参数有效微调的低资源视觉问答方法。
该方法通过对视觉语言预训练模型进行参数有效微调将预训练模型中蕴涵的知识迁移到下游视觉问答任务中,并以专家混合的方式增加参数有效模型的容量,提高了视觉问答模型在低资源场景下学习的稳定性和有效性。
在多个数据集不同低资源设置下的实验表明所提方法是现有唯一在所有数据集上性能都优于全模型微调的参数有效微调方法。

\end{enumerate}

\chinesekeywordstype{视觉问答;分布外泛化;鲁棒性;低资源学习;图网络;多模态预训练模型}{应用基础}

\end{chineseabstract}
