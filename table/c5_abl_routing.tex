\begin{table}[!t]
\caption{训练样本数$N_{\mathcal{D}}=64$时不同路由机制的性能比较 
% MixPHM-Token和MixPHM-Sent表示用 Fedus~等人~\cite{fedus2021switch}提出的分词级别和句子级别的随机路由策略为MixPHM选择激活的专家。 
% MixPHM-Rep是本章提出的基于表征聚合的路由策略。
% Time表示训练时每次参数更新的平均用时(单位:秒)。
}
\label{tab:c5_abl_routing}
\setlength{\tabcolsep}{7.65mm}{
\begin{tabularx}{\linewidth}{@{}lcccc}
\toprule
\multirow{1}{*}{Method} 
&Time (s)
&\multicolumn{1}{c}{VQA v2 val}
&\multicolumn{1}{c}{GQA test-dev}
&\multicolumn{1}{c}{OK-VQA test}
\\ 
\midrule

MixPHM-Token
&0.693
&47.67$~\std{0.92}$ 
&36.23$~\std{0.89}$ 
&17.77$~\std{0.89}$
\\

MixPHM-Sent
&0.683
&47.69$~\std{0.99}$ 
&36.13$~\std{0.86}$ 
&17.83$~\std{1.32}$
\\

MixPHM-Rep
&0.675
&48.00$~\std{0.95}$ 
&\textbf{36.77$~\std{0.55}$} 
&18.25$~\std{1.46}$
\\ 

MixPHM
&0.668
&\textbf{48.26$~\std{0.56}$}
&\text{36.75$~\std{0.55}$}
&\textbf{18.58$~\std{1.42}$}
\\ 

\bottomrule
\end{tabularx}
}
\end{table}
