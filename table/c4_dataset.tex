\begin{table}[!t]
\caption{
数据集的统计信息 
% Metric表示数据集提供的输入鲁棒性的评价指标,QType表示问题的类型,len(Q)表示问题的平均长度。
% \#IQ、\#Pert/CE 和 \#Ori/Easy分别表示图像-问题对的总数、添加了输入变化后样本的数量和原始样本数量。
}
\label{tab:c4_benchmark}
\setlength{\tabcolsep}{1.18mm}{
\begin{tabularx}{\textwidth}{@{}lccccccccc}
\toprule

\multirow{2}{*}{Dataset} 
&\multirow{2}{*}{Perturbation} 
&\multirow{2}{*}{Metric}
&\multirow{2}{*}{QType} 
&Training%\multicolumn{1}{c}{\multirow{2}{*}{\makecell{Training\\Dataset}}}
&\multicolumn{4}{c}{Evaluation} 
\\ 
\cmidrule(l){6-9}
& & &
&Dataset
&len(Q) &\#IQ &\#PER/CE &\#ORI/Easy 
\\ 
\midrule
% 换个说法说 
VQA-Rep~\cite{shah2019cycle} 
&Rephrasing 
&CS($m$) 
&All 
&VQA v2 train
&7.15 
&\eqmakebox[log][r]{162k} 
&\eqmakebox[log][r]{121,516} 
&\eqmakebox[log][r]{40,504} 
\\ 
% 改写 
VQA P2~\cite{whitehead2020learning} 
&Par\&Syn\&Ant 
&CS($m$) 
&All 
&VQA v2 train
&6.32 
&\eqmakebox[log][r]{52k} 
&\eqmakebox[log][r]{26,512} 
&\eqmakebox[log][r]{25,814}
\\ 

IV-VQA~\cite{agarwal2020towards} 
&Invariant object 
&\#flips 
&All 
&VQA v2 train
&5.85 
&\eqmakebox[log][r]{120k} 
&\eqmakebox[log][r]{83,700} 
&\eqmakebox[log][r]{36,181} 
\\ 

CV-VQA~\cite{agarwal2020towards} 
&Covariant object 
&\#flips 
&Num  
&VQA v2 train
&5.83 
&\eqmakebox[log][r]{4k} 
&\eqmakebox[log][r]{4,141} 
&\eqmakebox[log][r]{2,641}
\\ 

VQA-CE~\cite{dancette2021beyond} 
&Counterexample 
&- 
&All 
&VQA v2 train
&6.19 
&\eqmakebox[log][r]{214k} 
&\eqmakebox[log][r]{63,298} 
&\eqmakebox[log][r]{147,681}
\\ 
\bottomrule
\end{tabularx}
}
\end{table}