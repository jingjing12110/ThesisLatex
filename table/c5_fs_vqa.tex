\begin{table}[!t]
\caption{
与多模态少样本学习方法的性能对比 
% FewVLM是基于提示学习的全模型微调方法,MixPHM$^{\dagger}$表示使用MixPHM以参数有效的方式调节FewVLM模型。
}
\label{tab:c5_vqa_few_shot}
\setlength{\tabcolsep}{6.75mm}{
\begin{tabularx}{\linewidth}{@{}lccccc@{}}
\toprule
\multirow{1}{*}{Method} 
&Model Size 
&\#Param (\%)
&\multirow{1}{*}{VQAv2}
&\multirow{1}{*}{GQA} 
&\multirow{1}{*}{OK-VQA} 
\\

\midrule

Frozen~\cite{tsimpoukelli2021multimodal} 
&7B
&-
&38.2 &12.6 & -  
\\ 

PICa-Base~\cite{yang2022empirical} 
&175B 
&-
&\text{54.3} &\text{43.3} &- 
\\

PICa-Full~\cite{yang2022empirical} 
&175B 
&-
&\textbf{56.1} &\textbf{48.0} &- 
\\
% $_{base}$
FewVLM~\cite{jin2022good}  
&225M  
&100\% 
&48.2 
&32.2 
&15.0 
\\ 

MixPHM$^{\dagger}$
&226M
&0.39\%
&49.3
&33.4
&\textbf{19.2}
\\ 


\bottomrule
\end{tabularx}
}
\end{table}
