\begin{table}[!t]
\caption{训练样本数$N_{\mathcal{D}}=64$时不同正则的性能比较 
% ${\mathcal{L}_{\text{cs}}}$是由 Zuo~等人~\cite{zuo2022taming}提出的一种一致性正则。 
% ${\mathcal{L}^{\text{I}}_{\text{Ra}}}$ 和 ${\mathcal{L}^{\text{II}}_{\text{Ra}}}$ 分别表示本章提出的冗余正则$\mathcal{L}_{\text{Ra}}$的第一和第二项。
}
\label{tab:c5_abl_ra}
\setlength{\tabcolsep}{10.32mm}{
\begin{tabularx}{\linewidth}{lccc}
\toprule
\multirow{1}{*}{Method} 
&\multicolumn{1}{c}{VQA v2 val}
&\multicolumn{1}{c}{GQA test-dev}
&\multicolumn{1}{c}{OK-VQA test}
\\ 
\midrule

\rowcolor{gray!20}
Finetuning
&46.87$~\std{0.57}$
&34.22$~\std{0.59}$
&16.65$~\std{1.02}$
\\

MixPHM$^*$
&47.30$~\std{0.67}$ 
&34.66$~\std{0.78}$ 
&18.05$~\std{1.16}$ 
\\ 

\quad + ${\mathcal{L}_{\text{cs}}}$
&46.70$~\std{0.66}$ 
&34.83$~\std{1.35}$ 
&17.37$~\std{1.38}$ 
\\

\quad + ${\mathcal{L}^{\text{I}}_{\text{Ra}}}$
&47.42$~\std{0.71}$ 
&34.69$~\std{0.96}$ 
&18.25$~\std{1.54}$ 
\\

\quad + ${\mathcal{L}^{\text{II}}_{\text{Ra}}}$
&47.71$~\std{0.85}$ 
&36.10$~\std{0.83}$ 
&18.21$~\std{1.08}$
\\

\rowcolor{DeepSkyBlue!20}
\quad + $\mathcal{L}_{\text{Ra}}$
&\textbf{47.65$~_{\pm{0.56}}$}
&\textbf{36.75$~_{\pm{0.55}}$}
&\textbf{17.02$~_{\pm{1.70}}$}
\\ 

\bottomrule
\end{tabularx}
}
\end{table}
