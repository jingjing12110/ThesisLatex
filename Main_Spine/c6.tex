% !TeX root = ../main.tex

\xchapter{总结与展望}{Conclusion and Future Work}

% 本章首先对本文所进行的视觉问答的相关研究内容进行总结,然后在此基础上,对后续工作进行了展望。

\xsection{总结}{Conclusion}

视觉问答是视觉语言领域代表性的任务之一,因具有重要的理论研究价值及广阔的实际应用前景而成为了近几年的研究热点。尽管视觉问答在近年来已取得一系列突破性的进展,但目前计算机模型的表现与人类的表现还相差甚远,远远没有达到大规模普及和落地应用的水平。因此,本文从促进视觉问答模型实用性的角度出发,探讨视觉问答模型的多粒度问题、分布外泛化问题、输入鲁棒性问题和低资源学习问题,并提出了对应的解决方案。

本文主要研究内容总结如下:

1. 为应对视觉问答模型对视频中多粒度视觉语言信息利用不充分这一挑战,本文提出了基于多粒度视觉语言推理的视频问答方法。本文从视频理解的角度出发探讨一个完备的视频问答模型所需要具备的基本功能,在不同的粒度上表征视频的视觉和语言内容并在表征整合时兼顾表征的多样性,提出利用图统一视觉语言的多尺度编码及多样性表征的整合。该框架在不同粒度上表征视频视觉和语言内容,提高了对多粒度视觉语言信息的有效利用,而在保留表征判别性的同时自适应地学习不同表征的重要性,提升了跨模态表征融合的有效性。


2. 为应对语言偏见导致视觉问答模型泛化能力不足这一挑战,本文提出了基于图生成建模的视觉问答分布外泛化方法。本文将视觉问答的分布外泛化问题表述为组合泛化问题,利用域内数据来生成现有视觉概念的新组合来使视觉问答模型产生分布外的答案,提出了一种基于图生成建模的训练策略。该策略以一定的概率生成新的关系矩阵或新的节点表征来促使基线视觉问答模型泛化到分布外样本,并使用本文提出的梯度分布一致性损失来约束具有对抗性扰动的数据分布和生成的数据分布的梯度一致性,极大地缓解图生成建模过程中不稳定梯度的问题。


3. 为应对视觉问答模型对视觉语言输入变化鲁棒性不足这一挑战,本文提出了基于相关信息瓶颈理论的鲁棒视觉问答方法。本文从信息论的角度分析视觉问答模型输入鲁棒性差可能的原因进而发现提高这种鲁棒性本质上就是使模型获得的表征更加紧凑和更加任务相关。
基于该发现,本文提出了相关信息瓶颈理论,在为视觉问答任务微调视觉语言预训练模型时通过鼓励模型学习到的表征收敛到最小充分统计量显著地提高了基线视觉问答模型的输入鲁棒性。


4. 为应对视觉问答模型在低资源场景下学习能力不足这一挑战,本文提出了基于冗余感知参数有效微调的低资源视觉问答方法。本文通过专家混合的方式对大规模视觉语言预训练模型进行参数有效微调来将蕴含于预训练模型中丰富的知识迁移到给定的视觉问答中,并提出了冗余感知的参数有效微调方法。
该方法以专家混合的方式增加参数有效模型的容量,提高了视觉问答模型在低资源场景下学习的稳定性和有效性,是现有唯一在所有数据集上性能都优于全模型微调的参数有效微调方法。




\xsection{展望}{Perspective for Future Work}

近年来,视觉问答技术飞速发展,尤其是大规模视觉语言预训练技术的出现,将下游视觉问答任务在标准数据集上的性能提升到了人的水平。
然而,现有的视觉问答系统仍面临着如泛化能力弱、输入鲁棒性不足、少样本学习能力不强等本文探讨的问题,离实际应用还有很大差距。
因此,为推进视觉问答模型的实用和落地,视觉问答中还有许多值得探讨和研究的问题,本文对未来可能的相关研究进行如下简单展望:

% \begin{itemize}[wide,leftmargin=0pt,itemsep=1pt]

1. 提高视觉问答模型的对抗鲁棒性和可解释性。
近来,大规模视觉语言预训练技术虽然将下游视觉问答任务在标准数据集上的性能提升到了人的水平。
但这种优势也仅在标准数据集上能够保持,在面对更具挑战的真实视觉问答场景,现有的方法仍面临着泛化能力、鲁棒性差等实际问题。此外,因为缺乏对视觉语言模型及输出的可解释性,降低了人们对模型输出的结果的信赖。因此,提高视觉问答模型的鲁棒性和可解释性是实用视觉问答模型亟待解决的问题之一。
% 当人类回答问题时,会根据问题进行推理,寻找可以支持答案的证据。

2. 为生成式视觉问答构建更科学和合理的评测标准。
当前为了应对开放式视觉问答以及提高模型的泛化能力,最新基于视觉语言预训练的视觉问答研究~\cite{cho2021unifying,wang2022simvlm} 已经开始将视觉问答问题从原来的分类任务转换成文本生成任务,但对于生成式视觉问答模型的评测标准却没有改变。因此,如何更科学地评测生成式视觉问答有待继续研究。

3. 使用更少的监督训练信息。在大数据和互联网时代,每分钟都会产生大量的图像和视频等视觉媒体数据。然而,这些海量的媒体数据通常并未包含任何人工监督信息标注,或者仅含有极少量的人工监督信息标注。因此,设计能够在极少甚至无需监督信息的情况下运行的视觉问答模型,以便对互联网中的海量媒体数据进行问答,成为了一个紧迫的需求和挑战。

% 3. 探讨视觉问答任务背后的语言学问题。
% 当前大多数视觉问答工作都是从计算机视觉的角度出发的,更关注对视觉内容的理解。因而对视觉问答任务中涉及到的诸如语言歧义(Ambiguity)、语言模糊(Vagueness)等语言学问题的探讨相对较少。因此,如何发掘视觉问答任务背后的语言学问题,如何构建视觉内容与语义、句法和语用等语言学元素的交互有待继续研究。

% \item[(4)] \textit{将基于图像、视频的二维视觉问答推广到三维场景。} 
% 与图像、视频等二维视觉场景相比,三维视觉场景~\cite{yun2021pano,li2022learning}包含更加丰富的视觉信息,更符合真实的人机交互场景现状,是实现视觉问答落地的必经之路。因此,对三维视觉场景下的视觉问答的研究具有重要的研究意义和研究价值。
% % 可以辅助对整个视觉场景的感知和理解。

% \end{itemize}
